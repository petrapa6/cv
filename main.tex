%%%%%%%%%%%%%%%%%%%%%%%%%%%%%%%%%%%%%%%%%
% "ModernCV" CV and Cover Letter
% LaTeX Template
% Version 1.1 (9/12/12)
%
% This template has been downloaded from:
% http://www.LaTeXTemplates.com
%
% Original author:
% Xavier Danaux (xdanaux@gmail.com)
%
% License:
% CC BY-NC-SA 3.0 (http://creativecommons.org/licenses/by-nc-sa/3.0/)
%
% Important note:
% This template requires the moderncv.cls and .sty files to be in the same 
% directory as this .tex file. These files provide the resume style and themes 
% used for structuring the document.
%https://github.com/petrapa6/cv/blob/master/about/paragraph.md
%%%%%%%%%%%%%%%%%%%%%%%%%%%%%%%%%%%%%%%%%

\documentclass[11pt,a4paper,sans]{moderncv} % Font sizes: 10, 11, or 12; paper sizes: a4paper, letterpaper, a5paper, legalpaper, executivepaper or landscape; font families: sans or roman
\usepackage{standalone}
\moderncvstyle{classic} % CV theme - options include: 'casual' (default), 'classic', 'oldstyle' and 'banking'

\definecolor{color0}{rgb}{0,0,0}% black
\definecolor{color1}{cmyk}{1, 0.43, 0, 0}% light blue
\definecolor{color2}{cmyk}{1, 0.43, 0, 0}% dark grey

\usepackage{enumitem}
\setlist{nosep}

\usepackage[backend=bibtex,defernumbers=true,style=ieee,sortcites=true,citestyle=numeric-comp]{biblatex}
\addbibresource{main.bib}

\renewcommand{\labelitemi}{\textcolor{color1}{\raisebox{.45ex}{\rule{.8ex}{.8ex}}}}
\renewcommand{\labelitemii}{\textcolor{color1}{\raisebox{.45ex}{\rule{.8ex}{.8ex}}}}
\renewcommand{\labelitemiii}{\textcolor{color1}{\raisebox{.45ex}{\rule{.8ex}{.8ex}}}}
\renewcommand{\labelitemiv}{\textcolor{color1}{\raisebox{.45ex}{\rule{.8ex}{.8ex}}}}

\defbibenvironment{refbibenv}
{\itemize[label=\textcolor{color1}{\raisebox{.45ex}{\rule{.8ex}{.8ex}}},leftmargin=2.53cm,labelsep=0.33cm]}
{\enditemize}
{\small\item}
% \renewcommand*{\bibfont}{\Font}

\usepackage{lipsum} % Used for inserting dummy 'Lorem ipsum' text into the template

\usepackage[scale=0.85]{geometry} % Reduce document margins
%\setlength{\hintscolumnwidth}{3cm} % Uncomment to change the width of the dates column
%\setlength{\makecvtitlenamewidth}{10cm} % For the 'classic' style, uncomment to adjust the width of the space allocated to your name

%\usepackage[utf8]{inputenc}

%\usepackage{booktabs}
\usepackage{fontawesome}
\usepackage{marvosym} % For cool symbols.
%\usepackage{hyperref}

\linepenalty=1000

%----------------------------------------------------------------------------------------
%	NAME AND CONTACT INFORMATION SECTION
%----------------------------------------------------------------------------------------

\firstname{Pavel} % Your first name
\familyname{Petráček} % Your last name

% All information in this block is optional, comment out any lines you don't need
% \title{3rd year Ph.D. student}
\address{Czech Technical University In Prague}{Department of Cybernetics}{Multi-Robot Systems group}
\email{petrapa6@fel.cvut.cz} 
\mobile{+420\,739\,757\,519}

\homepage{http://mrs.felk.cvut.cz/members/phdstudents/pavel-petracek}{mrs.felk.cvut.cz/pavel-petracek}

% social link \faGithub, \faSkype, \faLinkedin,\faStackExchange, and \faStackOverflow
\extrainfo{
    % \mobilesymbol(+420) 739 757 519\\
    \faGithub\href{https://github.com/petrapa6}{ GitHub} \quad
    \faGoogle\href{https://scholar.google.com/citations?user=IwzN6MQAAAAJ}{ Google Scholar}\\
    % \faLinkedin\href{https://www.linkedin.com/abc/}{ Linkedin} \quad
    % \faSkype\href{https://skype.com/abc}{Skype}
    }



%\social[linkedin][www.linkedin.com]{name}
% The first argument is %the url for the clickable link, the second argument is the url displayed in the %template - this allows special characters to be displayed such as the tilde in this %example

\photo[90pt][0.3pt]{fig/portrait} % The first bracket is the picture height, the second is %the thickness of the frame around the picture (0pt for no frame)
%\quote{Not Attention, Patience is all we need.}

%----------------------------------------------------------------------------------------

\newcommand{\cvdoublecolumn}[2]{%
  \cvitem[.75em]{}{%
    \begin{minipage}[t]{\listdoubleitemcolumnwidth}#1\end{minipage}%
    \hfill%
    \begin{minipage}[t]{\listdoubleitemcolumnwidth}#2\end{minipage}%
    }%
}



% \usepackage{multibbl}
\newcommand\Colorhref[3][color1]{\href{#2}{\small\color{#1}#3}}


% \newcommand{\cvreference}[7]{%
%     \textbf{#1}\newline% Name
%     \ifthenelse{\equal{#2}{}}{}{\addresssymbol~#2\newline}%
%     \ifthenelse{\equal{#3}{}}{}{#3\newline}%
%     \ifthenelse{\equal{#4}{}}{}{#4\newline}%
%     \ifthenelse{\equal{#5}{}}{}{#5\newline}%
%     \ifthenelse{\equal{#6}{}}{}{\emailsymbol~\texttt{#6}\newline}%
%     \ifthenelse{\equal{#7}{}}{}{\phonesymbol~#7}}

\begin{document}

\makecvtitle % Print the CV title


%----------------------------------------------------------------------------------------
%	Personal information
%----------------------------------------------------------------------------------------

\section{Personal information}
\cventry{Nationality}{}{}{Czech}{}{}
\cventry{Date of birth}{}{}{November 26, 1994}{}{}
\cventry{Languages}{}{}{Czech (mothertongue), great conversational English}{}{}
% \cventry{Work address}{}{}{Karlovo namesti 13, 121 35 Prague 2}{}{}

%----------------------------------------------------------------------------------------
%	EDUCATION SECTION
%----------------------------------------------------------------------------------------

\section{Education}

\cventry{2019--present}{Doctoral candidate in Informatics, }{}{}{Department of Cybernetics, Faculty of Electrical Engineering, Czech Technical University in Prague (FEE CTU)}
{ --- \textbf{Ph.D. topic}: Robust UAV localization in perception-degraded environments\\
  --- \textbf{supervisor}: doc. Ing. Martin Saska, Dr. rer. nat.\\
  --- \textbf{publication count} (since 2019): 13 impacted journals (+ 1 preprint), 3 conference proceedings\\
  --- \textbf{h-index}: 7 in WoS, 11 in Google Scholar, \textbf{citations count}: 126 in WoS, 388 in Google Scholar}

\cventry{2017--2019}{Ing. --- Master degree in Cybernetics and robotics, }{}{}{FEE CTU}{}
% {Finished with self-contained system for documenting interiors of historical buildings with an autonomous UAV}
\cventry{2014--2017}{Bc. --- Bachelor degree in Cybernetics and robotics, }{}{}{FEE CTU}{}
% {Finished with at-the-time novel research on decentralized swarming of UAV teams in obstacle-filled environments}

\section{Experience}

\vspace{2mm}
\cventry{2019--present}{Doctoral candidate and research fellow at Multi-Robot Systems research group, }{}{}{FEE CTU}{%
  --- \textbf{research:} lightweight yet robust autonomy of mobile robots in perceptually degraded environments | decentralized swarming systems | robustness maximization in aerial robotics\\
  --- \textbf{key responsibilities:} co-development of \Colorhref{https://github.com/ctu-mrs/mrs_uav_system}{MRS UAV system} | transferring research concepts to the real world | design and realization of robotic experiments | robotic competitions | demos for investors, industrial partners, students, and media | student supervision | field popularization | academic teaching | summer school and workshop organization}

\vspace{2mm}
\subsection{Research projects \& competitions}
\vspace{2mm}
\cventry{2020--2022}{\Colorhref{http://mrs.felk.cvut.cz/projects/darpa}{DARPA Subterranean Challenge}: }{}{}{Exploration of unknown subterranean environments with a cooperative team of ground and aerial autonomous robots}{
  --- \textbf{contributions \& responsibilities:} development of novel methods of lightweight perception, localization, and mapping of UAVs in perception-degraded environments | UAV system design | real-time systems integration | system evaluation and testing | key member for in situ deployment of aerial robots}
\cventry{2018--2022}{\Colorhref{http://mrs.felk.cvut.cz/ram2022dronument}{Dronument}: }{}{}{Documentation of interiors of historical structures with autonomous aerial robots}{--- \textbf{contributions \& responsibilities:} development of a robust HW \& SW system capable of deploying a fully autonomous UAV team within interiors of historical structures | focus on on-board UAV localization and prevention of its degeneracy in geometrically featureless environments | deployment of the system for documenting 17 historical objects (including 2 UNESCO sites) with direct use in heritage preservation}
\cventry{2017-present}{\Colorhref{https://www.youtube.com/watch?v=Ax3ONfo1hMA}{Swarming}: }{}{}{Decentralized communication-less control of UAVs among obstacles}{--- \textbf{contributions:} novel bio-inspired algorithms for communication-less perception-aware coordination of UAV teams in environments with obstacles}
\cventry{2020-present}{\Colorhref{https://www.youtube.com/watch?v=QHpifXJzH5g}{DOFEC}: }{}{}{Extinguishment of fires in aboveground floors using an autonomous UAV}{--- \textbf{contributions:} detection and localization of fires from on-board sensors | mission planning}
% \cventry{2017-present}{Decentralized swarming}{}{}{}{}

\vspace{2mm}
\subsection{International stays}
\vspace{2mm}
\cventry{2023}{Autonomous Robots Lab at NTNU: }{}{}{2 months research stay, cooperation on doctoral topic with prof. Kostas Alexis}{}

\subsection{Industry}
\vspace{2mm}

\cventry{2023--present}{Fly4Future s.r.o.: }{}{technical consulting | grant writing | employee training}{}{}
\cventry{2016-2017}{CertiCon a.s.: }{}{learned how to properly think about and write automated software tests | gained experience in corporate project management and scheduling}{}{}
\cventry{2012-2014}{KD planeta s.r.o.: }{}{first-hand experience with robotic automation --- interaction between human operators, robotic manipulators, and CNC machinery}{}{}

\section{Honors \& awards}

\vspace{2mm}
\cventry{2022}{Methodology M17+: }{}{}{excellent international evaluation of our \Colorhref{https://mrs.felk.cvut.cz/dg18p02ovv069-fvz}{\textbf{Dronument functional sample}}}{}
\cventry{2021}{DARPA Subterranean Challenge: }{}{}{team CTU-CRAS-NORLAB competing with international universities and companies (e.g., Caltech, MIT, ETH Zürich) in multi-robot search~\&~rescue operations in underground environments}{--- 1st place among non-funded teams in the Urban Circuit, real-world deployment (\$500k)\\--- 2nd place among all teams in the Final Round, virtual deployment (\$500k)}
\cventry{2019}{Dean's price for astounding Master thesis at FEE CTU}
{}{}{}{--- \textbf{topic:} Design, localization and position control of a specialized UAV platform for documentation of historical monuments}
\cventry{2017}{Dean's price for astounding Bachelor thesis at FEE CTU}
{}{}{}{--- \textbf{topic:} Decentralized model of a swarm behavior Boids in ROS}%\\--- contributions: Novel research on decentralized control of a UAV team in obstacle-filled environments. Results published in journal Bioinspiration \& Biomimetics.}
 
\section{Academic activities}

\cvitem{Teaching}{%
  \vspace{-0.6em}
  \begin{itemize}
    \item Algorithms and Programming: Python and basic programming algorithms for Bachelor students
    \item Multi-Robot Aerial Systems: for Master students, \Colorhref{https://cw.fel.cvut.cz/wiki/_media/courses/mrs/tutorials/task_03_swarm.pdf}{link} to example task 
  \end{itemize}
}
\vspace{-3mm}

% \cvitem[1em]{Workshop presentations}{\textbf{Workshop presentations:}}
\cvitem{Workshops}{%
  \vspace{-0.6em}
  \begin{itemize}
    \item Seminar tasks introduction, \textit{In IEEE RAS Summer School on Multi-Robot Systems}, 2022.
    \item \Colorhref{http://mrs.felk.cvut.cz/dronumentworkshop}{Dronument workshop} (organizer and speaker), \textit{hosted at FEE CTU}, 2021.
    \item Importance Sampling: Degradation-Aware Alternative to Voxelization in Robot Pose Estimation, \textit{In IEEE IROS \Colorhref{https://ippc-iros23.github.io/index.html}{IPPC} \textbf{and} \Colorhref{https://sites.google.com/view/ropem/}{ROPEM} workshops}, 2023.
    \item Cooperative UAV Autonomy of Dronument: New Era in Cultural Heritage Preservation, \textit{In IEEE IROS \Colorhref{https://ippc-iros23.github.io/index.html}{IPPC} workshop}, 2023.
    \item Decentralized Aerial Swarms Using Vision-Based Mutual Localization, \textit{In IEEE IROS (Workshop on Integrated Perception, Planning, and Control for Physically and Contextually-Aware Robot Autonomy)}, 2018.
  \end{itemize}
}

\vspace{-3mm}
% \cvitem{}{\textbf{Committee at conference proceedings:}}
\cvitem{Conference committee}{%
  \vspace{-0.6em}
  \begin{itemize}
    \item Co-chair of session \textit{Micro and Mini UAS I} at ICUAS'22 (chair: prof. Subodh Bhandari).
  \end{itemize}
}
\vspace{0.5em}
\cvitem{Reviewer for journals and conferences}{%
  \vspace{-0.6em}
  \begin{itemize}
    \item Transactions on Cybernetics
    \item Transactions on Robotics (T-RO)
    \item Robotics and Automation Letters (RA-L)
    \item International Conference on Robotics and Automation (ICRA)
    \item International Conference on Intelligent Robots and Systems (IROS)
  \end{itemize}
}

% \bibliographystyle{journal}{plainyrrev}
% \nocite{journal}{*}
% \bibliography{journal}{journal}
% {\large \textsc{Refereed Journal Articles}}
% \subsection{Communicated Journal Article}
% \cventry{2020}{Pratik Dutta, Aditya Prakash Patra, and Sriparna Saha}{}{DeePROG: An Attention based Deep Multi-modal Architecture for Disease Gene Prognosis}{In \textit{IEEE Transactions on Biomedical Engineering}}{}

% \subsection{In Conference Proceedings}
% \newbibliography{conference}
% \nocite{conference}{*}
% \bibliographystyle{conference}{plainyrrev}
% \bibliography{conference}{conference}
% {\large \textsc{Refereed Conference Publications}}

%----------------------------------------------------------------------------------------
%	WORK EXPERIENCE SECTION
%----------------------------------------------------------------------------------------

% \section{Research Experience}
% \subsection{Indian Institute of Technology, ABC}
% \cventry{June,2019 -- present}{\textit{Identifying Protein-protein Interaction from Biomedical text}}{}{}{}
% {Developing a deep multi-modal architecture for accurately predicting protein interaction information from biomedical text. 
% }
% \cvitem{Advisor :}{\textbf{Dr. abc xyz}, \textit{Associate Professor, Department of Computer Science \& Engineering}, IIT abc ({\Colorhref{https://www.personal_webpage.com/} {\textit{Personal Web-page}}})}

% \cventry{July,2018 -- present}{\textit{Developing Deep Multi-modal Architecture for Biomedical Problems}}{}{}{}
% {Analyzing different modalities of genes like gene expression profiles, protein 3D structure, underlying amino acid sequence using popular deep learning models to obtain deeper insight into the underlying biological system. 
% }
% \cvitem{Advisor :}{\textbf{Dr. abc xyz}, \textit{Associate Professor, Department of Computer Science \& Engineering}, IIT abc ({\Colorhref{https://www.personal_webpage.com/} {\textit{Personal Web-page}}})}

% \subsection{Indian Institute of Technology, XYZ}
% \cventry{January,2015 -- Dec,2015}{\textit{Design and Synthesis of Reversible Multi-dimentional Nearest-Neighbour(NN) Quantum Circuit}}{}{}{}{Proposed an approach for designing and physically implementing of the multi-dimensional quantum circuits maintaining nearest-neighbor complacency that use minimal number of SWAP gates.}
% \cvitem{Advisor :}{\textbf{Dr. abc xyz}, \textit{Associate Professor, Department of Computer Science \& Engineering}, IIT abc ({\Colorhref{https://www.personal_webpage.com/} {\textit{Personal Web-page}}})}


% \cventry{2012 -- 2013}{\textit{Text Document Clustering with Semantic Similarity through Wordnet}}{}{}{}{Improvement of the text document clustering task over conventional methods by introducing WORDNET and some better clustering algorithms.}
% \cvitem{Advisor :}{\textbf{Dr. abc xyz}, \textit{Associate Professor, Department of Computer Science \& Engineering}, IIT abc ({\Colorhref{https://www.personal_webpage.com/} {\textit{Personal Web-page}}})}



%----------------------------------------------------------------------------------------
%	Academic achievements
%----------------------------------------------------------------------------------------

% \section{Academic Achievements \& Recognitions }

% \cvitem{2018}{\textbf{Session Chair of the session "Prediction"} in \textbf{$25^{th}$ \textit{International Conference of Neural Information Processing} (ICONIP 2018)}, Siem Reap, Cambodia.}

% \cvitem{2018}{Invited to conduct lab sessions in \textit{\textbf{"Training Program on Machine Learning For Ocean Acoustics and Climate Data Analysis"}}, during 22-36 October 2018 at \textbf{Defence R\&D Organization- Naval Physical \& Oceanographic Laboratory (DRDO-NPOL), Kochi, Kerala}.}

%----------------------------------------------------------------------------------------
%	COMPUTER SKILLS SECTION
%----------------------------------------------------------------------------------------

% \section{Computer skills}

% \cvitem{Programming Languages}{Python, PyTorch, keras, R, C, C++, Advanced JAVA}
% \cvitem{Web Technologies}{HTML 5, PHP, JSP, Javascript}
% \cvitem{Database}{SQL, MySQL, Apache, Neo4j}

\section{Supervised students}

\cventry{Ing.}{Vojtěch Nydrle, }{}{}{Cybernetics and robotics, FEE CTU}{--- thesis: Extinguishing of Indoor Fires by an Autonomous UAV}

\cventry{Bc.}{Vojtěch Nydrle, }{}{}{Cybernetics and robotics, FEE CTU}{--- thesis: Design of a specialized UAV platform for the discharge of a fire extinguishing capsule (Dean's price for astounding Bachelor thesis)}
\cventry{}{Martin Fischer, }{}{}{Cybernetics and robotics, FEE CTU}{--- thesis: Lidar and multi-camera calibration and fusion (Dean's price for astounding Bachelor thesis)}

\section{Peer-reviewed publications}

\vspace{2mm}
\subsection{Journal articles}

\vspace{0.2cm}
\begin{refcontext}
  \nocite{*}
  \printbibliography[keyword={first\_author},keyword={journal},notkeyword={citation},heading=none,title={},env=refbibenv]
  \printbibliography[keyword={second\_author},keyword={journal},notkeyword={citation},heading=none,title={},env=refbibenv]
  \printbibliography[keyword={third\_author},keyword={journal},notkeyword={citation},heading=none,title={},env=refbibenv]
  \printbibliography[notkeyword={first\_author},notkeyword={second\_author},notkeyword={third\_author},keyword={journal},notkeyword={citation},heading=none,title={},env=refbibenv]
\end{refcontext}

\subsection{Conference articles}

\vspace{0.2cm}
\begin{refcontext}
  \nocite{*}
  \printbibliography[keyword={mine},keyword={conference},notkeyword={citation},heading=none,title={},env=refbibenv]
\end{refcontext}

% \section{Position of Responsibility}
% \cventry{2016-2020}{Executive member of IEEE Student Branch}{}{IIT ABC}{}{}
% \cventry{April 1-5, 2019}{Organizer, GIAN Workshop on subjects}{}{IIT ABC}{}{}
%----------------------------------------------------------------------------------------

% \section{Teaching Assistantship}
% \cventry{Fall, 2019 :}{CS564: Foundations of Machine Learning}{}{IIT ABC}{}{}
% \cventry{Spring, 2019 :}{CS342: Operating System Lab}{}{IIT ABC}{}{}
% \cventry{Fall, 2018 :}{CS564: Foundations of Machine Learning}{}{IIT ABC}{}{}

% \section{Referees}

% \begin{tabular}{lr}
% % Referee 1
% \begin{minipage}[t]{3in}
% \textbf{doc. Ing. Martin Saska, Dr. rer. nat.}\\
% \textit{Department of Cybernetics, FEE CTU} \\
% \textit{Faculty of Electrical Engineering}\\
% Czech Technical University in Prague\\
% \Letter\ \href{mailto:martin.saska@fel.cvut.cz}{martin.saska@fel.cvut.cz}
% \end{minipage}
% \end{tabular}

\end{document}
