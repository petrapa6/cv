%%%%%%%%%%%%%%%%%%%%%%%%%%%%%%%%%%%%%%%%%
% "ModernCV" CV and Cover Letter
% LaTeX Template
% Version 1.1 (9/12/12)
%
% This template has been downloaded from:
% http://www.LaTeXTemplates.com
%
% Original author:
% Xavier Danaux (xdanaux@gmail.com)
%
% License:
% CC BY-NC-SA 3.0 (http://creativecommons.org/licenses/by-nc-sa/3.0/)
%
% Important note:
% This template requires the moderncv.cls and .sty files to be in the same 
% directory as this .tex file. These files provide the resume style and themes 
% used for structuring the document.
%https://github.com/petrapa6/cv/blob/master/about/paragraph.md
%%%%%%%%%%%%%%%%%%%%%%%%%%%%%%%%%%%%%%%%%

\documentclass[11pt,a4paper,sans]{moderncv} % Font sizes: 10, 11, or 12; paper sizes: a4paper, letterpaper, a5paper, legalpaper, executivepaper or landscape; font families: sans or roman
\usepackage{standalone}
\moderncvstyle{classic} % CV theme - options include: 'casual' (default), 'classic', 'oldstyle' and 'banking'

\definecolor{color0}{rgb}{0,0,0}% black
\definecolor{color1}{cmyk}{1, 0.43, 0, 0}% light blue
\definecolor{color2}{cmyk}{1, 0.43, 0, 0}% dark grey

\usepackage{enumitem}
\setlist{nosep}

\usepackage[backend=bibtex,defernumbers=true,style=ieee,sortcites=true,citestyle=numeric-comp]{biblatex}
\addbibresource{main.bib}

\renewcommand{\labelitemi}{\textcolor{color1}{\raisebox{.45ex}{\rule{.8ex}{.8ex}}}}
\renewcommand{\labelitemii}{\textcolor{color1}{\raisebox{.45ex}{\rule{.8ex}{.8ex}}}}
\renewcommand{\labelitemiii}{\textcolor{color1}{\raisebox{.45ex}{\rule{.8ex}{.8ex}}}}
\renewcommand{\labelitemiv}{\textcolor{color1}{\raisebox{.45ex}{\rule{.8ex}{.8ex}}}}

\defbibenvironment{refbibenv}
{\itemize[label=\textcolor{color1}{\raisebox{.45ex}{\rule{.8ex}{.8ex}}},leftmargin=2.53cm,labelsep=0.33cm]}
{\enditemize}
{\small\item}
% \renewcommand*{\bibfont}{\Font}

\usepackage{lipsum} % Used for inserting dummy 'Lorem ipsum' text into the template

\usepackage[scale=0.85]{geometry} % Reduce document margins
%\setlength{\hintscolumnwidth}{3cm} % Uncomment to change the width of the dates column
%\setlength{\makecvtitlenamewidth}{10cm} % For the 'classic' style, uncomment to adjust the width of the space allocated to your name

%\usepackage[utf8]{inputenc}

%\usepackage{booktabs}
\usepackage{fontawesome}
\usepackage{marvosym} % For cool symbols.
%\usepackage{hyperref}

\linepenalty=1000

%----------------------------------------------------------------------------------------
%	NAME AND CONTACT INFORMATION SECTION
%----------------------------------------------------------------------------------------

\firstname{Pavel} % Your first name
\familyname{Petráček} % Your last name

% All information in this block is optional, comment out any lines you don't need
% \title{3rd year Ph.D. student}
\address{Czech Technical University In Prague}{Department of Cybernetics}{Multi-Robot Systems group}
\email{petrapa6@fel.cvut.cz} 
% \mobile{}
 
%\social{github}{stefano-bragaglia}

\homepage{http://mrs.felk.cvut.cz/members/phdstudents/pavel-petracek}{mrs.felk.cvut.cz/pavel-petracek}

% social link \faGithub, \faSkype, \faLinkedin,\faStackExchange, and \faStackOverflow
\extrainfo{
    % \mobilesymbol(+420) 739 757 519\\
    \faGithub\href{https://github.com/petrapa6}{ GitHub} \quad
    \faGoogle\href{https://scholar.google.com/citations?user=IwzN6MQAAAAJ}{ Google Scholar}\\
    % \faLinkedin\href{https://www.linkedin.com/abc/}{ Linkedin} \quad
    % \faSkype\href{https://skype.com/abc}{Skype}
    }



%\social[linkedin][www.linkedin.com]{name}
% The first argument is %the url for the clickable link, the second argument is the url displayed in the %template - this allows special characters to be displayed such as the tilde in this %example

\photo[90pt][0.3pt]{fig/portrait} % The first bracket is the picture height, the second is %the thickness of the frame around the picture (0pt for no frame)
%\quote{Not Attention, Patience is all we need.}

%----------------------------------------------------------------------------------------

\newcommand{\cvdoublecolumn}[2]{%
  \cvitem[.75em]{}{%
    \begin{minipage}[t]{\listdoubleitemcolumnwidth}#1\end{minipage}%
    \hfill%
    \begin{minipage}[t]{\listdoubleitemcolumnwidth}#2\end{minipage}%
    }%
}



% \usepackage{multibbl}
\newcommand\Colorhref[3][color1]{\href{#2}{\small\color{#1}#3}}


% \newcommand{\cvreference}[7]{%
%     \textbf{#1}\newline% Name
%     \ifthenelse{\equal{#2}{}}{}{\addresssymbol~#2\newline}%
%     \ifthenelse{\equal{#3}{}}{}{#3\newline}%
%     \ifthenelse{\equal{#4}{}}{}{#4\newline}%
%     \ifthenelse{\equal{#5}{}}{}{#5\newline}%
%     \ifthenelse{\equal{#6}{}}{}{\emailsymbol~\texttt{#6}\newline}%
%     \ifthenelse{\equal{#7}{}}{}{\phonesymbol~#7}}

\begin{document}

\makecvtitle % Print the CV title




%----------------------------------------------------------------------------------------
%	EDUCATION SECTION
%----------------------------------------------------------------------------------------

\section{Education}

\cventry{2019--present}{Pursuing Ph.D. in Informatics}{}{}{Department of Cybernetics, Faculty of Electrical Engineering, Czech Technical University in Prague (FEE CTU)}
{--- \textbf{general research on:} lightweight yet robust localization and mapping of mobile robots in perception-degraded environments, decentralized swarming systems, robustness maximization in aerial robotics\\
--- 8 publications in impacted journals and 3 contributions to conference proceedings since 2019\\
--- \textbf{h-index}: 3 in WoS, 6 in Google Scholar, \textbf{citations count}: 27 in WoS, 76 in Google Scholar (39 in 2021)\\
--- supervisor: doc. Ing. Martin Saska, Dr. rer. nat.}

\cventry{2017--2019}{Ing. (= Master of Science), Cybernetics and robotics}{}{}{FEE CTU}{}
% {Finished with self-contained system for documenting interiors of historical buildings with an autonomous UAV}
%\cvitem{CGPA :}{7.96/10}
\cventry{2014--2017}{Bc. (= Bachelor of Science), Cybernetics and robotics}{}{}{FEE CTU}{}
% {Finished with at-the-time novel research on decentralized swarming of UAV teams in obstacle-filled environments}

\section{Experience}

\cventry{2019--present}{Research fellow at Multi-Robot Systems group}{}{}{FEE CTU}{--- \textbf{responsibilities:} general research, co-development of \Colorhref{https://github.com/ctu-mrs/mrs_uav_system}{MRS UAV system}, transferring research ideas into the real-world (design and realization of robotic experiments, participation in robotic competitions), robotic demos for investors, students and public media, supervision of students, popularization of the university and the field, academic courses \& summer school preparations, workshop organization}

\subsection{Research projects \& competitions}
\cventry{2020--2022}{DARPA Subterranean Challenge}{}{}{Exploration of unknown subterranean environments with a cooperative team of ground and aerial autonomous robots}{--- \textbf{contributions \& responsibilities:} Development of novel methods of lightweight perception, localization, and mapping of UAVs in perception-degraded environments, UAV system design, real-time systems integration, system evaluation and testing, key member for in situ deployment of aerial robots. Research published in several journal publications.}
\cventry{2018--present}{Dronument}{}{}{Documentation of interiors of historical structures with autonomous aerial robots}{--- \textbf{contributions \& responsibilities:} Development of a robust HW \& SW system capable of deploying a fully autonomous UAV team within interiors of historical structures. Focus on on-board UAV localization and prevention of its degeneracy in geometrically featureless environments. Deployment of the system for documenting 17 historical objects (including Chateau Kromeriz from the UNESCO heritage database) with direct use for heritage preservation. Research published in several journal publications.}
\cventry{2017-present}{Swarming}{}{}{Decentralized control of UAV teams in obstacle-filled environments}{--- \textbf{contributions:} Novel bio-inspired algorithms for communication-less perception-aware coordination of UAV teams in environments with obstacles. Research published in several academic publications.}
\cventry{2020-present}{DOFEC}{}{}{Extinguishment of fires in aboveground floors using an autonomous UAV}{--- \textbf{contributions:} detection and localization of fires from on-board sensors, mission planning}
% \cventry{2017-present}{Decentralized swarming}{}{}{}{}

\subsection{Industry}

\cventry{2016-2017}{Software testing}{CertiCon a.s.}{Learned how to properly think about and write automated software tests. Gained experience in corporate project management and scheduling}{}{}
\cventry{2012-2014}{Robotic automation}{KD planeta s.r.o.}{First-hand experience with robotic automation --- interaction between human operators, robotic manipulators, and CNC machinery}{}{}

\section{Honors \& awards}

\cventry{2021}{DARPA Subterranean Challenge}{}{}{Part of team CTU-CRAS-NORLAB competing with well-known foreign universities and companies (e.g., Caltech, MIT, ETH Zürich) in multi-robot search~\&~rescue operations in underground environments}{--- 1st place among non-funded teams in the Urban Circuit, real-world deployment (\$500k)\\--- 2nd place among all teams in the Final Round, virtual deployment (\$500k)}
\cventry{2019}{Dean's price for astounding Master thesis}
{}{}{FEE CTU. Related to the Dronument project}{--- topic: Design, localization and position control of a specialized UAV platform for documentation of historical monuments\\}
\cventry{2017}{Dean's price for astounding Bachelor thesis}
{}{}{FEE CTU}{--- topic: Decentralized model of a swarm behavior Boids in ROS\\--- contributions: Novel research on decentralized control of a UAV team in obstacle-filled environments. Results published in journal Bioinspiration \& Biomimetics.}


\section{Peer-reviewed publications}

\subsection{Journal articles}

\vspace{0.2cm}
\begin{refcontext}[sorting=none]
  \nocite{*}
  \printbibliography[keyword={mine},keyword={first_author},keyword={journal},notkeyword={citation},heading=none,title={},env=refbibenv]
  \printbibliography[keyword={mine},keyword={second_author},keyword={journal},notkeyword={citation},heading=none,title={},env=refbibenv]
  \printbibliography[keyword={mine},keyword={third_author},keyword={journal},notkeyword={citation},heading=none,title={},env=refbibenv]
  \printbibliography[keyword={mine},notkeyword={first_author},notkeyword={second_author},notkeyword={third_author},keyword={journal},notkeyword={citation},heading=none,title={},env=refbibenv]
\end{refcontext}
  
\subsection{Conference articles}

\vspace{0.2cm}
\begin{refcontext}[sorting=none]
  \nocite{*}
  \printbibliography[keyword={mine},keyword={conference},notkeyword={citation},heading=none,title={},env=refbibenv]
\end{refcontext}

\section{Secondary academic activities}

\cvitem{}{\textbf{Workshop presentations:}}
\cvitem{}{\begin{itemize}
  \item Decentralized Aerial Swarms Using Vision-Based Mutual Localization, \textit{In IEEE IROS (Second Workshop on Multi-robot Perception-Driven Control and Planning)}, 2018.
\end{itemize}}
\cvitem{}{\textbf{Referee for journals and conference proceedings:} Transactions on Cybernetics, Robotics and Automation Letters, International Conference on Robotics and Automation, International Conference on Intelligent Robots and Systems.}

% \bibliographystyle{journal}{plainyrrev}
% \nocite{journal}{*}
% \bibliography{journal}{journal}
% {\large \textsc{Refereed Journal Articles}}
% \subsection{Communicated Journal Article}
% \cventry{2020}{Pratik Dutta, Aditya Prakash Patra, and Sriparna Saha}{}{DeePROG: An Attention based Deep Multi-modal Architecture for Disease Gene Prognosis}{In \textit{IEEE Transactions on Biomedical Engineering}}{}

% \subsection{In Conference Proceedings}
% \newbibliography{conference}
% \nocite{conference}{*}
% \bibliographystyle{conference}{plainyrrev}
% \bibliography{conference}{conference}
% {\large \textsc{Refereed Conference Publications}}

%----------------------------------------------------------------------------------------
%	WORK EXPERIENCE SECTION
%----------------------------------------------------------------------------------------

% \section{Research Experience}
% \subsection{Indian Institute of Technology, ABC}
% \cventry{June,2019 -- present}{\textit{Identifying Protein-protein Interaction from Biomedical text}}{}{}{}
% {Developing a deep multi-modal architecture for accurately predicting protein interaction information from biomedical text. 
% }
% \cvitem{Advisor :}{\textbf{Dr. abc xyz}, \textit{Associate Professor, Department of Computer Science \& Engineering}, IIT abc ({\Colorhref{https://www.personal_webpage.com/} {\textit{Personal Web-page}}})}

% \cventry{July,2018 -- present}{\textit{Developing Deep Multi-modal Architecture for Biomedical Problems}}{}{}{}
% {Analyzing different modalities of genes like gene expression profiles, protein 3D structure, underlying amino acid sequence using popular deep learning models to obtain deeper insight into the underlying biological system. 
% }
% \cvitem{Advisor :}{\textbf{Dr. abc xyz}, \textit{Associate Professor, Department of Computer Science \& Engineering}, IIT abc ({\Colorhref{https://www.personal_webpage.com/} {\textit{Personal Web-page}}})}

% \subsection{Indian Institute of Technology, XYZ}
% \cventry{January,2015 -- Dec,2015}{\textit{Design and Synthesis of Reversible Multi-dimentional Nearest-Neighbour(NN) Quantum Circuit}}{}{}{}{Proposed an approach for designing and physically implementing of the multi-dimensional quantum circuits maintaining nearest-neighbor complacency that use minimal number of SWAP gates.}
% \cvitem{Advisor :}{\textbf{Dr. abc xyz}, \textit{Associate Professor, Department of Computer Science \& Engineering}, IIT abc ({\Colorhref{https://www.personal_webpage.com/} {\textit{Personal Web-page}}})}


% \cventry{2012 -- 2013}{\textit{Text Document Clustering with Semantic Similarity through Wordnet}}{}{}{}{Improvement of the text document clustering task over conventional methods by introducing WORDNET and some better clustering algorithms.}
% \cvitem{Advisor :}{\textbf{Dr. abc xyz}, \textit{Associate Professor, Department of Computer Science \& Engineering}, IIT abc ({\Colorhref{https://www.personal_webpage.com/} {\textit{Personal Web-page}}})}



%----------------------------------------------------------------------------------------
%	Academic achievements
%----------------------------------------------------------------------------------------

% \section{Academic Achievements \& Recognitions }

% \cvitem{2018}{\textbf{Session Chair of the session "Prediction"} in \textbf{$25^{th}$ \textit{International Conference of Neural Information Processing} (ICONIP 2018)}, Siem Reap, Cambodia.}

% \cvitem{2018}{Invited to conduct lab sessions in \textit{\textbf{"Training Program on Machine Learning For Ocean Acoustics and Climate Data Analysis"}}, during 22-36 October 2018 at \textbf{Defence R\&D Organization- Naval Physical \& Oceanographic Laboratory (DRDO-NPOL), Kochi, Kerala}.}

%----------------------------------------------------------------------------------------
%	COMPUTER SKILLS SECTION
%----------------------------------------------------------------------------------------

% \section{Computer skills}

% \cvitem{Programming Languages}{Python, PyTorch, keras, R, C, C++, Advanced JAVA}
% \cvitem{Web Technologies}{HTML 5, PHP, JSP, Javascript}
% \cvitem{Database}{SQL, MySQL, Apache, Neo4j}

\section{Supervised students}

\cventry{Bc.}{Vojtěch Nydrle}{}{}{Cybernetics and robotics, FEE CTU}{--- thesis: Design of a specialized UAV platform for the discharge of a fire extinguishing capsule\\--- awarded with the Dean's price for astounding Bachelor thesis}
\cventry{}{Martin Fischer}{}{}{Cybernetics and robotics, FEE CTU}{--- thesis: Lidar and multi-camera calibration and fusion\\--- awarded with the Dean's price for astounding Bachelor thesis}


% \section{Position of Responsibility}
% \cventry{2016-2020}{Executive member of IEEE Student Branch}{}{IIT ABC}{}{}
% \cventry{April 1-5, 2019}{Organizer, GIAN Workshop on subjects}{}{IIT ABC}{}{}
%----------------------------------------------------------------------------------------

% \section{Teaching Assistantship}
% \cventry{Fall, 2019 :}{CS564: Foundations of Machine Learning}{}{IIT ABC}{}{}
% \cventry{Spring, 2019 :}{CS342: Operating System Lab}{}{IIT ABC}{}{}
% \cventry{Fall, 2018 :}{CS564: Foundations of Machine Learning}{}{IIT ABC}{}{}

% \section{Referees}

% \begin{tabular}{lr}
% % Referee 1
% \begin{minipage}[t]{3in}
% \textbf{doc. Ing. Martin Saska, Dr. rer. nat.}\\
% \textit{Department of Cybernetics, FEE CTU} \\
% \textit{Faculty of Electrical Engineering}\\
% Czech Technical University in Prague\\
% \Letter\ \href{mailto:martin.saska@fel.cvut.cz}{martin.saska@fel.cvut.cz}
% \end{minipage}
% \end{tabular}

\end{document}
